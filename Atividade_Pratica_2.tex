\documentclass{article}
\usepackage[utf8]{inputenc}
\usepackage{hyperref}

\title{Atividade Prática 2}
\author{Bruna Bicalho }
\date{Novembro 2021}

\begin{document}

\maketitle

\section{Realizar os seguintes experimentos:}

\subsection{Qual o ip local da sua máquina?}

Usei o comando ipconfig e o  ip local da máquina é: 192.168.1.118
\paragraph{}

C:\Users\brbicalh2101>ipconfig

Windows IP Configuration


Ethernet adapter Ethernet:

   Media State . . . . . . . . . . . : Media disconnected
   Connection-specific DNS Suffix  . :

Ethernet adapter Ethernet 5:

   Media State . . . . . . . . . . . : Media disconnected
   Connection-specific DNS Suffix  . :

Ethernet adapter Ethernet 2:

   Media State . . . . . . . . . . . : Media disconnected
   Connection-specific DNS Suffix  . :

Wireless LAN adapter Conexão Local* 9:

   Media State . . . . . . . . . . . : Media disconnected
   Connection-specific DNS Suffix  . :

Wireless LAN adapter Conexão Local* 10:

   Media State . . . . . . . . . . . : Media disconnected
   Connection-specific DNS Suffix  . :

Wireless LAN adapter Wi-Fi:

   Connection-specific DNS Suffix  . :
   IPv4 Address. . . . . . . . . . . : 192.168.1.118
   Subnet Mask . . . . . . . . . . . : 255.255.255.0
   Default Gateway . . . . . . . . . : 192.168.1.254

Tunnel adapter Teredo Tunneling Pseudo-Interface:

   Connection-specific DNS Suffix  . :
   IPv6 Address. . . . . . . . . . . : 2001:0:2877:7aa:2833:fbf6:36e2:e97d
   Link-local IPv6 Address . . . . . : fe80::2833:fbf6:36e2:e97d%14
   Default Gateway . . . . . . . . . : ::



\subsection{Qual o ip local da macalan?}

Conectei a macalan (a partir do ssh bbicalho@ssh.inf.ufpr.br) e o ip dela é (usando ifconfig): 10.17.110.6 

\paragraph{}

bbicalho@macalan:~$ ifconfig
ens19: flags=4163<UP,BROADCAST,RUNNING,MULTICAST>  mtu 1500
        inet 10.17.110.6  netmask 255.255.255.0  broadcast 10.17.110.255
        inet6 fe80::216:3eff:fe73:6  prefixlen 64  scopeid 0x20<link>
        ether 00:16:3e:73:00:06  txqueuelen 1000  (Ethernet)
        RX packets 687607  bytes 2014108942 (2.0 GB)
        RX errors 0  dropped 0  overruns 0  frame 0
        TX packets 664995  bytes 878070587 (878.0 MB)
        TX errors 0  dropped 0 overruns 0  carrier 0  collisions 0

eth0: flags=4163<UP,BROADCAST,RUNNING,MULTICAST>  mtu 1500
        inet 200.17.202.6  netmask 255.255.255.128  broadcast 200.17.202.127
        inet6 2801:82:80ff:8001:216:3eff:fe79:6  prefixlen 64  scopeid 0x0<global>
        inet6 fe80::216:3eff:fe79:6  prefixlen 64  scopeid 0x20<link>
        ether 00:16:3e:79:00:06  txqueuelen 1000  (Ethernet)
        RX packets 50527107  bytes 44654271808 (44.6 GB)
        RX errors 0  dropped 45  overruns 0  frame 0
        TX packets 40906100  bytes 61555641655 (61.5 GB)
        TX errors 0  dropped 0 overruns 0  carrier 0  collisions 0

lo: flags=73<UP,LOOPBACK,RUNNING>  mtu 65536
        inet 127.0.0.1  netmask 255.0.0.0
        inet6 ::1  prefixlen 128  scopeid 0x10<host>
        loop  txqueuelen 1000  (Loopback Local)
        RX packets 82330  bytes 328117828 (328.1 MB)
        RX errors 0  dropped 0  overruns 0  frame 0
        TX packets 82330  bytes 328117828 (328.1 MB)
        TX errors 0  dropped 0 overruns 0  carrier 0  collisions 0


\subsection{Qual a rota padrão da sua máquina? }
Para definir a rota padrão da máquina, usei o comando route PRINT.
\paragraph{}
C:\Users\brbicalh2101>route PRINT
\paragraph{}
===========================================================================
Interface List
 10...2c ea 7f e5 e3 c3 ......Realtek USB GbE Family Controller
 11...2c ea 7f e5 e3 c2 ......Intel(R) Ethernet Connection (10) I219-LM
 12...54 b8 38 8f 48 06 ......Check Point Virtual Network Adapter For Endpoint VPN Client
 20...bc 17 b8 6f d7 d2 ......Microsoft Wi-Fi Direct Virtual Adapter
 19...be 17 b8 6f d7 d1 ......Microsoft Wi-Fi Direct Virtual Adapter #2
 17...8e aa f5 a7 e1 67 ......Intel(R) Wi-Fi 6 AX201 160MHz
  1...........................Software Loopback Interface 1
 14...00 00 00 00 00 00 00 e0 Microsoft Teredo Tunneling Adapter
===========================================================================

IPv4 Route Table
===========================================================================
Active Routes:
Network Destination        Netmask          Gateway       Interface  Metric
          0.0.0.0          0.0.0.0    192.168.1.254    192.168.1.118     50
        127.0.0.0        255.0.0.0         On-link         127.0.0.1    331
        127.0.0.1  255.255.255.255         On-link         127.0.0.1    331
  127.255.255.255  255.255.255.255         On-link         127.0.0.1    331
      192.168.1.0    255.255.255.0         On-link     192.168.1.118    306
    192.168.1.118  255.255.255.255         On-link     192.168.1.118    306
    192.168.1.255  255.255.255.255         On-link     192.168.1.118    306
        224.0.0.0        240.0.0.0         On-link         127.0.0.1    331
        224.0.0.0        240.0.0.0         On-link     192.168.1.118    306
  255.255.255.255  255.255.255.255         On-link         127.0.0.1    331
  255.255.255.255  255.255.255.255         On-link     192.168.1.118    306
===========================================================================
Persistent Routes:
  None

IPv6 Route Table
===========================================================================
Active Routes:
 If Metric Network Destination      Gateway
 14    331 ::/0                     On-link
  1    331 ::1/128                  On-link
 14    331 2001::/32                On-link
 14    331 2001:0:2877:7aa:2833:fbf6:36e2:e97d/128
                                    On-link
 14    331 fe80::/64                On-link
 14    331 fe80::2833:fbf6:36e2:e97d/128
                                    On-link
  1    331 ff00::/8                 On-link
 14    331 ff00::/8                 On-link
===========================================================================
Persistent Routes:
  None

\subsection{Qual a rota padrão da macalan? }
Para definir a rota padrão da macalan usei o comando route -n.
\paragraph{}
bbicalho@macalan:~$ route -n
\paragraph{}
Tabela de Roteamento IP do Kernel
Destino         Roteador        MáscaraGen.    Opções Métrica Ref   Uso Iface
0.0.0.0         200.17.202.62   0.0.0.0         UG    0      0        0 eth0
10.17.110.0     0.0.0.0         255.255.255.0   U     0      0        0 ens19
10.254.0.0      200.17.202.3    255.255.0.0     UG    0      0        0 eth0
169.254.0.0     0.0.0.0         255.255.0.0     U     1000   0        0 ens19
200.17.202.0    0.0.0.0         255.255.255.128 U     0      0        0 eth0


\subsection{Qual o caminho (route) mais comum entre sua máquina e a macalan? }
Para definir o caminho mais comum entre a minha máquina e a macalan usei o comando tracert e o endereço da ufpr.
\paragraph{}
C:\Users\brbicalh2101>tracert inf.ufpr.br
\paragraph{}
Tracing route to inf.ufpr.br [200.17.202.3]
over a maximum of 30 hops:

  1     3 ms     3 ms     3 ms  www.webgui.Nokiawifi.com [192.168.1.254]
  2    25 ms    24 ms    17 ms  201-4-133-1.user.veloxzone.com.br [201.4.133.1]
  3    38 ms    98 ms    53 ms  100.122.84.53
  4    54 ms    49 ms    51 ms  100.122.19.63
  5    76 ms    75 ms    78 ms  100.122.25.171
  6    74 ms    71 ms    71 ms  100.122.19.96
  7   113 ms    92 ms    96 ms  as1916.saopaulo.sp.ix.br [187.16.220.208]
  8    95 ms    94 ms    93 ms  200.143.255.143
  9   108 ms   101 ms   122 ms  200.143.254.130
 10    91 ms    95 ms    92 ms  p2-v103-araucaria-lapa.pop-pr.rnp.br [200.238.139.10]
 11    91 ms    98 ms    86 ms  200.17.202.62
 12    94 ms    93 ms    97 ms  urquell.c3sl.ufpr.br [200.17.202.3]

Trace complete.


\subsection{A partir de diferentes máquinas, o caminho (route) até a macalan muda?}
Sim.

\paragraph{}
\paragraph{}
\section{Responder as perguntas}

\subsection{Por que faltam camadas no roteador e no switch do slide 7?}

A função do roteador é apenas passar a informação de um lado para o outro, por isso não tem outras camadas como aplicação. Já o switch é um dispositivo que simplesmente conecta todos os elementos da rede. Ele atua como ponte ou unidade de controle para que computadores, impressoras, servidores e todos os outros tipos dispositivos possam se comunicar.

\subsection{Por que seu roteador wi-fi não é um roteador de verdade?}
Porque ele roteia apenas para um local. Assim, o nome correto para o “roteador wi-fi de casa” seria ponto de acesso cuja função é se conectar com o roteador do provedor de Internet. 

\subsection{Qual a porta padrão dos seguintes protocolos: DHCP, HTTPS e POP3?}

\begin{itemize}
\item 67/UDP: BOOTP (BootStrap Protocol) server; também utilizada por DHCP (Protocolo de configuração dinâmica do Host)
\item 68/UDP: BOOTP client; também utilizada por DHCP
\item 443/TCP: HTTPS - HTTP Protocol over TLS/SSL (transmissão segura)(Camada de transporte seguro)
\item 110/TCP: POP3 (Post Office Protocol version 3): Protocolo de Correio Eletrônico, versão 3 - usada para recebimento de e-mail
\item 995/TCP: POP3 sobre SSL (transmissão segura)
\end{itemize}

Fonte: \url{https://pt.wikipedia.org/wiki/Lista_de_portas_dos_protocolos_TCP_e_UDP}

\subsection{Ainda há endereços IPv4 disponíveis no Brasi? Quando esgotaram ou quando esgotarão?}
O estoque de endereços IPv4 para a região da América Latina e o Caribe esgotou-se em 19/8/2020. 
A partir desta data, “as organizações que venham a solicitar justificadamente a necessidade de endereços IPv4, e que ainda não contaram com alocação desse recurso, será dada a opção de permanecer em uma fila de pedidos aprovados. Estes pedidos serão eventualmente atendidos de acordo com os recursos que venham a se tornar disponíveis após processos de recuperação e devolução” (IPV6.BR).

Fonte: \url{https://ipv6.br/post/fim-do-ipv4/}


\end{document}
